\abstract{Language, a widely used mean of communication, dynamic, robust and still 
so simple; a specific human capacity, capable of carrying our thoughts and maybe 
the only feature that make us humans fundamentally different from other species, 
and still so vaguely understood. Approximately from 3000 to 7000 languages are spoken nowadays, 
all of them hold remarkable distinctions one from another, but still have much in common. 
The purpose of this work is to analyze languages under a statistical point of view. 
It is argued that languages are self-organizing systems, and that language usage creates 
and shapes what languages are. The linguistic competence of a speaker is attributed to 
self-organization phenomena, but not to a nativist hypothesis. In order to carry out 
such analysis, we use information collected from various languages in the UCLA Phonological 
Segment Inventory Database. This database is a statistical survey of the phoneme inventories 
in 451 languages of the world. For the purpose of analyzing how a specific speech inventory 
is used in a certain language, we also use a corpus of texts collected in public databases 
to create a corpus of phonological segments, by means of a pronouncing dictionary. 
Here we used the English pronouncing dictionary provided by The Carnegie Mellon University. 
The concomitant usage of this information, in a statistical analysis, 
might resemble what spoken languages really are. The statistical study of a language is important 
to understand how it works and structures itself, how the contrastive elements are used to build 
words and utterances, and how those contrasts are established. The analysis may be used to create 
the basis to explain certain language changes observed during its evolution. 
We review some quantitative aspects on language, Zipf's law, Zipf-Mandelbrot's law, Menzarath's law,
Heap's law. These laws were used to stablish an analysis on different levels and to investigate
the relationship between them. Language, as a mean of communication, was analyzed under
the information theory perspective. 
%In order to better understand how language, as 
%a communication mean, works and evolves, it is important to statistically analyze it.
%It is also of great significance to 
%perform a comparative analysis of various languages in order to understand each one of them and to 
%understand what is the common ground on the way languages are structured. The statistical analysis 
%interlanguage and intralanguage might shed some light on this still so cloudy subject.}

\renewcommand\abstractname{Resumo} 
\begin{abstract} 
Linguagem, uma forma de comunicação amplamente utilizada, dinâmica, robusta e ainda assim tão simples; 
uma faculdade específica dos humanos, capaz de levar nossos pensamentos e talvez a única característica 
que nos distinga de outras espécies; e ainda tão pouco compreendida. Aproximadamente de 3000 a 7000 línguas são 
faladas nos dias atuais, todas possuem diferenças marcantes em relação às outras, mas ainda assim possuem 
muito em comum. O propósito deste trabalho é realizar uma análise comparativa de línguas a partir 
de um ponto de vista estatístico. Defende-se que as línguas são sistemas auto-organizativos, e que o próprio 
uso da linguagem cria e molda o que elas são. Atribui-se a competência linguística de um falante 
a um fenômeno auto-organizativo, ao invés de uma hipótese inata. Uma primeira análise é feita utilizando-se 
informações de diversas línguas coletadas no banco de dados \textit{UCLA Phonological Segment Inventory Database}. 
Este banco de dados é um levantamento estatístico do inventário fonêmico de 451 línguas do mundo. 
Para uma análise de como um determinado inventário é utilizado em uma língua, utilizamos um corpus de textos 
coletados em bases de dados públicas para criar um corpus de segmentos fonológicos, através da utilização de um 
dicionário de pronúncia. Utilizamos aqui o dicionário de pronúncia do inglês fornecido pela 
\textit{Universidade de Carnegie Mellon}. A utilização concomitante destas informações, em uma análise estatística, 
assemelha-se ao que as línguas faladas realmente são. O estudo estatístico de uma língua é importante para 
entender o seu funcionamento e estruturação, como os elementos contrastivos são utilizados na construção de 
palavras e sentenças, e como se estabelecem esses contrastes. Uma análise dos dados permite determinar fundamentos 
capazes de explicar observações a cerca da evolução de uma língua. 
Fazemos uma revisão de aspectos quantitativas da linguagem, lei de Zipf, lei de Zipf-Mandelbrot, lei de Menzerath e
lei de Heaps. Estas leis são utilizadas em uma análise em diferentes níveis e a relação entre elas é investigada.
Sob a ótica da Teoria da Informação é feita uma análise da Linguagem enquanto meio de comunicação.
%Para entender como a linguagem funciona e evolui, enquanto um meio de comunicação, é importante analisá-la 
%estatisticamente.
%Estabelecer um comparativo entre estatísticas 
%de várias línguas é importante para o entendimento de cada uma delas e para o entendimento do que há em comum nas 
%formas como as línguas se estruturam. A utilização de uma análise interlíngua e intralíngua torna, assim, mais claro 
%o entendimento deste assunto ainda nebuloso.
% desenvolver
%

%Este trabalho apresenta uma análise comparativa de diversas línguas a partir de um ponto de vista estatístico, utilizando uma abordagem auto-organizativa que busca explicar as maneiras pelas quais os sons da fala são agrupados.

\end{abstract} 
