\chapter{Conclusions}

The understanding of language origin and evolution requires a deep study
and identification of what are the universal features that drive language evolution.
We adopt the usage-based theory of language in which its cognitive organization
is a product of the speaker experience, rather than a set of abstract rules and structures
that indirectly relates to the language experience. Grammar is based on usage, contains
many details of probabilities of occurrence and co-occurrence, and creates a dynamic network
from categorized instances of language use.
Frequency of use and its consequent cognitive organization have impact on language 
structure and usage. Speed of access related to token frequency, priming, the process 
of grammaticalization and the morphological and phonic characteristics of high and low frequency
words are examples of the impact frequency has on language and cognition \citep{bybee2001,bybee2003,bybee2007frequency,ellis2002}.
Studies \citep{saffran1996b,Saffran1999,Saffran2003} show that human track patterns and statistical regularities in artificial
grammars even when there is no correspondence to meaning or communicative intentions.

The idea of `universals of grammar' may be seen as a by-product of the emergent phenomena
of grammar, what is dependent on idiosyncratic localized interactions and the
process of spreading and shifting through populations and time.
From the basic principle that usage is a driven factor to shape language, we 
expect to observe common patterns and regularities in different languages.
The UPSID database is very important to analyze the type of segments used by different
languages in the world to build their speech segments repertoire. From the observation
of the data in this database we could realize that some speech segments are very
common among languages, while others are quite rare. The phone \textipa{[m]}, for example,
happens in 94.2\% of the languages in that database, while 427 speech sounds appear just once
among all 425 languages.

It is a common place to use more consonants than vowels (90\% of the languages 
use at most 6 times more consonants than vowels). The median number of vowels used by
languages is 4, and the median number of consonants is about 18.
We could observe that just a few phones have many co-occurring segments in their 
languages and are also found in different languages. It could give us a hint that 
they work like wildcards. Although some results might suggest that there is a binding 
force between voice and unvoiced counterparts, specifically, the existence of
an unvoiced consonant would make it more probable the existence of its voiced 
counterpart (not the other way around), it seems this is false, since 
the probability of a voiced segment given its voiceless counterpart is
0.406, and the probability of a voiceless segment given its voiced counterpart
is 0.415. Moreover, the frequency of occurrence of voiced and unvoiced segments
in UPSID is not so different. Voiced segments appear on 3.8\% of the languages
and unvoiced on 3.1\%. Another feature is that languages that use fewer 
segments to build their speech repertoire tend to use mostly common segments,
and languages that use many speech segments, tend to use rare phones to
build their repertoire.

%UPSID
%existence of an unvoiced consonant makes it more probable the existence of it
%voiced counterpart then the other way around
%(check)
%mean P(voiced|unvoiced) = 0.406
%mean P(unvoiced|voiced) = 0.415
%unvoiced consonants are more common that voiced consonants
%(check)
%mean(freqvoiced)/425 =  0.038159
%mean(frequnvoiced)/425 =  0.031266
%std(freqvoiced) =  49.415
%std(frequnvoiced) =  42.555
%language that uses fewer segments in its repertoire --> common segments are used in the segments repertoire
%UPSID occurrence of a voiced stop facilitate the occurrence of its unvoiced counterpart in that language

Our first analysis consisted on observing the segments used by various languages around the
world to build their speech repertoire. As we understand language as a dynamic self-organized
system, we believe it is important to investigate how those segments are used, how they
build larger structures, what are the statistical characteristics of languages.
We believe that there might resemblances on the way languages use their segments and in the way
they structure themselves through usage. This type of analysis would require a great amount
of work, and therefore we have restricted our analysis to the linguistic patterns
observed and their mathematical models. Those patterns observed across different languages
or phenomena are also called \emph{laws}.

The ubiquity of scaling laws in nature suggests that complex systems organize themselves,
they are not just a collection of random independent realizations, but exhibit similar patterns
and relations that are observed in different scales. ``Scaling laws (...) pervade neural,
behavioral and linguistic activities''\citep{kello2010}. The ubiquity of normal distribution in 
nature is also well known. Although in the past it was not satisfactorily understood, it became clear
when the central limit theorem showed that random independent variables always combine to
produce, in the limit, a normal distribution. Complex systems are more than mere collections
of independent random variables, they have interdependent effects and behaviours, and therefore
might not be described by random distributions.

Scaling laws are also present in many cognitive activities. 
Steven's power law \citep{stevens1957} proposes a relation
between the magnitude of a physical stimulus and its perceived intensity, which are related 
by a power law in the form $\Psi(I) = k I^\alpha$, where $I$ is the magnitude of the physical
stimulus, $\alpha$ is the relating exponent and $k$ is the proportionality constant (which depends
on the type of the stimulation). Steven's power law is also known as a 
generalization of Weber-Fechner law to a wider range of sensations.
This law shows the existence of similar relations among variables at different
scales, which means that there is a scale invariance in perception.
%Concerning the motor control, \cite{lacquaniti1983} presented a power law relating 
%the angular velocity and the curvature when considering evidences from the execution
%of handwriting and drawing movements. 
These observed scaling laws are hypothesized 
to reflect the maximization of sensitivity and dynamic range of sensory systems \citep{copelli2007}
In the case of motor control systems, it is argued to be used to minimize the effects
of errors due to noise in the motor system \citep{harris1988}. The idea of a time dynamic scaling
memory retrieval, proposed by \cite{gordon2007}, is also consistent with the ubiquitous 
observations of scaling laws in cognition.

Some researchers claim that Zipf's law is shallow, since it is a result of choosing 
rank as the independent variable and for that reason even random generated texts present 
a Zipf pattern. However we have presented some arguments claiming that the intermittent silence
model is shallow itself and it does not hold all features present is a natural communication
process. A visual analysis of the log-log plots of frequency of occurrence versus rank creates
a first evidence. The Zipf model presupposes a monotonically decreasing sequence of values
for the frequency of occurrences, however for a random generated text, all words with the 
same length share the same probability of occurrence, creating a stair case pattern.
This is not observed in natural texts, expect for the poorly sampled words, which
appear at the end of the tail of the decreasing curve.
Another aspect that makes natural and random texts distinct is evidenced by
the histogram of words length. 
Random generated text present a geometrically distributed frequency of occurrence
for words length. The same is also present when considering the number of words for a given length.
%words which probabilities are a exponentially decreasing function of it length. 
The number of words, as well as the frequency of occurrence of words, suffer a 
steeper decrease when approaching very short word lengths. A maximum is found for words
which length is equal to 7 letters, or 5 phones, or 1 syllable, when regarding the
number of existing words; or a maximum frequency of occurrence is found for letters 
of length equals to 3 letters, or 2 phones, or 1 syllable (depending on your unit
adopted to measure a words length). Random texts present an exponentially decreasing
number of words as the words length increase. The same is observed when regarding 
the frequency of occurrence. The plots of natural texts also show a much steeper 
decrease in the number of words and its frequency as the length increases, in comparison 
with random texts. 

We have simulated the creation of random texts, generating random symbols. We used 
three different approaches: uniform distributed symbols; symbols with the same
probabilities as what is observed in natural texts; symbols generated by a Markov
model with the same transitions probabilities from the natural text. Only the
random text created by a Markov model gets close to the observed pattern of
2-grams and 3-grams in real texts. The Markov model was also able to produce
words with the same behaviour found in natural texts, a higher word length around
2-3 symbols long words.

The inverse Zipf plot is another clue leading to the fact
that random texts do not present true Zipf distribution. 
The occurrence of \emph{hapax legomenon} on random texts is 
greater than in natural texts.
More over, random texts seems to present a more ragged pattern in the region
of high occurrence words, 
%Still rearing the inverse Zipf plot, it is also shown that 
``real texts fill the lexical spectrum much more efficiently and
regardless of the word length, suggesting that the meaningfulness of Zipf’s law is high''\citep{cancho2002}.
Other arguments are presented, such as the evaluation of consistency of ranks, 
indicating that Zipf's law might in fact be a fundamental law in natural languages
\citep{cancho2009,cancho2010}.

The exponent parameter in the Zipf distribution is important in determining the
statistical properties of that distribution. In order to compare data that follow
a power law relation, we need to find the best fit to the data. We have presented 
the usage of the maximum likelihood approach to determine the exponent parameters
that gives the best fit. We have derived the equations which lead to a root finding
problem, for which the solution is the maximum likelihood estimated parameter.
The same procedure is used to find the exponent and flattening parameters of
a Zipf-Mandelbrot model. This procedure was tested using real text data
and the results show a good performance. 
 

A definition of a distance measure between two different phones was proposed, 
consisting on the number of distinctive feature not share between these phones.
This definition is in accordance with the natural class definition proposed by \cite{flemming2005}.
We argue that phones are chosen to built higher order structures according to their
resemblances and distinctions. We analyzed the formation of triphones according
to this distance measure and we concluded that, more frequently, the triphones
are build using phones that have an intermediate distance between them. 
Too much similarity between consecutive phones must be avoid, in order to create
easy distinction between them, and so too much disparity must also be avoided,
to ensure easy in the articulation. This reflects once again the trade off
between the speaker effort and the listener effort. We have also observed that
the first pair of phones in a triphone tends to present more distinctiveness than the
second pair of phones.

Menzerath's law states a relation on the usage of tiles to create a whole.
According to this law, the length of the whole is inversely proportional
%As we considered the use of tiles in the formation of the whole, we also analyzed 
%the Menzerath's law, which states that the length of the whole is inversely proportional
to the length of the parts. We considered the duration of words utterance to estimate
an average duration in the pronunciation of word's parts (syllables and phones). 
The results show that as the word length (number of syllables of phones) increases,
the average duration of the parts decreases. We also observe this type of behaviour as
we analyze the relation between sentences length (number of words) and words length 
(number of syllables or letters).

As we consider language as a mean to communicate, exchange information, it is important
to analyse it under the information theory point of view. We want to characterize a source, 
which produces symbols according to a Zipf distribution, according to the information
content generated. We used the same steps proposed by \cite{grignetti} to derive an estimate
of the entropy of such a source. The exponent parameter of the distribution was shown
to be a sensitive parameter to characterize the source regarding its information production.
The estimates obtained, when compared with real data are shown to be slightly different,
what shows that the proposed approximation is good.
Small discrepancies are explained by numerical errors, small deviation on the
estimated distribution parameters, and the undersampling on the data, which is
represented as a staircase pattern on the low frequency region.

\cite{cristelli2012} proposes that Zipfian distributed data hold coherence among them.
We present the distortion created when the data from a certain range of ranks are 
extracted from the group. This process may create a distortion on the frequency-rank
relation. There will be no distortion if the extraction process itself only on high
rank types. There will be distortion on the low rank types if low rank types are
extracted. The extraction of the very high frequency types is called 
as the New York's effect. \cite{cristelli2012} argues that when there is coherence
among data, you may gather different data and still the Zipf's law will still hold.
But if you gather data from different sources with distinct characteristics,
even if they are coherent by themselves, when they are added, the new global data 
will not be coherent itself, and the Zipf's law will not hold anymore.
A simple example of the concatenation of different random data, created by 
different probability distributions, show that the same pattern is also observed
after the random data is concatenated. This could contradict what was expected 
by \cite{cristelli2012}, or could corroborate that random data are not truly Zipfian
distributed, and the Zipf like pattern was once mode generated as a byproduct of
selecting rank as the independent variable.

\cite{vanLeijenhorst} shows that a source which is distributed according to a 
generalized Zipf's law will present a Heap's law behaviour, there will be a 
sublinear relation between the text length and the vocabulary size.
We present a calculation to show that any source with any distribution will
present a monotonically growing lexicon size which converges to the 
underlying lexicon size. The sublinear lexical growth pattern is exemplified 
using a few examples from the real world.

In order to analyze the patterns of a language under different scales, analyzing
the occurrence of phones and formation of higher order structures, we believe
the usage of the Feature Theory might be providential since it give us a
direct and formal way to quantify a phone characteristics making it possible
to create a distance measure between two phones. We propose to measure the 
distance between two phones by the number of distinctive features not shared 
by them. By means of a multidimensional scaling we create a representation
of the phones in a vectorial space where the given linear distance will
provide the best match with the proposed distance measure. 
The graphical representation shows that this measure might be a meaningful choice.
We believe that its usage concomitantly with frequency of occurrence might
give us a better hint on the structure and organization of languages.




