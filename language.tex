\chapter{Language}
\epigraph{But the most noble and profitable invention of all others was that of speech ... whereby men register their thoughts, recall them when they are past, and also declare them to one another for mutual utility and conversation; without which there had been amongst men neither commonwealth, nor society, nor contract, nor peace, no more than amongst lions, bears and wolves.}{Thomas Hobbes (1651), Leviathan.}

Language refers to the forms of communication among people, or a human capacity for acquiring and using a complex system
of communication. Various means might be used to achieve communication, such as gesture, facial expressions, written text, etc.
The most usual is the spoken realization of language, what will be our main concern here.
% as a spoken form of communication among individuals. 
Since centuries ago, the human faculty of language has intrigued and motivated the studies seeking to
understand what is a language and how it works.
%Although the philosophical question `what is language and how it works?' has been settled centuries ago, 
%the major efforts on this subject date from the last century, and the only certain conclusion so far is 
%that language is an astonishing phenomena that rises as a communication process among humankind. 
All efforts made so far have brought until these days only a fragmentary and superficial 
insight of the communication phenomena. 
It is important to realize that, language is not just a collection of words that happen
to occur in succession, it is the grammar that express how we put the words together in
order to convey utterance and meaning. 
Although not yet understood, it is so simple that nearly every child 
masters it almost unconsciously.
The complex set of rules and patterns that creates spoken communication is
mastered by any individual.
%and still it is not fully understood by the language scientists.
%As pointed out by \cite{chomsky1969}, 
``A grammar of a language purports to be a description of the ideal 
speaker-hearer's intrinsic competence'' \cite{chomsky1969}
and still it is not fully understood by the language scientists. 
%The purpose of language is a meaningful communication. 
%In spite of the efforts to understand the mechanisms of communication, the binds and the meanings are still much vague.

Language is intrinsically bound to thought. Our thoughts are stated under our language knowledge, 
and we use it as a tool to express them. 
Our linguistic reality, the categories and usage in language, is said to shape how we perceive the world and the way we think. 
That is called `the Sapir-Whorf hypothesis', named after the American linguists Edward Sapir and Benjamin Lee Whorf. 

\begin{quote}
Human beings do not live in the objective world alone, nor alone in the world of social activity as ordinarily understood, 
but are very much at the mercy of the particular language which has become the medium of expression for their society. 
It is quite an illusion to imagine that one adjusts to reality essentially without the use of language and that language 
is merely an incidental means of solving specific problems of communication or reflection. The fact of the matter is that 
the `real world' is to a large extent unconsciously built upon the language habits of the group. No two languages are ever 
sufficiently similar to be considered as representing the same social reality. The worlds in which different societies live 
are distinct worlds, not merely the same world with different labels attached... We see and hear and otherwise experience 
very largely as we do because the language habits of our community predispose certain choices of interpretation. \citep{sapir1929}
\end{quote}



Comparing languages, it is possible to notice that the categorization of the world is different across cultures. 
\cite{whorf1940}, in a popular paper, refereed to Eskimo languages as having distinct categorizations 
for types of snow and therefor different words were used.
%Taking the English example, 
``We find that the idea of \textit{water} is expressed in a great variety of forms: 
one term serves to express water as a \textit{liquid}; another one, water in the form of a large expanse (\textit{lake});
others, water as running in a large body or in a small body (\textit{river} and \textit{brook}); 
still other terms express water in the form of \textit{rain}, \textit{dew}, \textit{wave}, and \textit{foam}. 
It is perfectly conceivable that this variety of ideas, each of which is expressed by a single independent term in English, 
might be expressed in other languages by derivations from the same term. Another example of the same kind, the words for 
\textit{snow} in Eskimo, may be given. Here we find one word, aput, expressing \textit{snow on the ground}; 
another one, qana, \textit{falling snow}; a third one, piqsirpoq, \textit{drifting snow}; and a fourth one, 
qimuqsuq, \textit{a snowdrift}''\citep{boas1911}.


Most of the languages have their own ways to express colors and numbers, but the existence of cross-linguistic universals 
categories is contested. \cite{regier2003} present a recent analysis of color categorization among languages of
industrialized and nonindustrialized societies. The results suggest that color categories may not be
universal. \cite{everett2005} described a language called Pirahã, spoken by 
an indigenous people of Amazonas, Brazil, which lacks such fine categorization, all they count on is the distinction 
between dark and bright (for colors), few and many (for numerals). 
Although they lack the semantic representation of these concepts, they appear to understand that there are different concepts. 
`A total lack of exact quantity language did not prevent the Pirahã from accurately performing a task which relied on 
the exact numerical equivalence of large sets'\citep{frank2008}.
Even if there are different categorizations in a culture, it might not be reflected on their 
expression as a language. Also, the existence of different words to acknowledge \textit{snow} in various
contexts doesn't imply that Eskimo are more concerned about snow than then the residents of Alaska who speak English.
%It is a difficult task to establish to what extent are language and thought bound together.
The extent to which language and thought are bound together is fuzzy and unclear.


Consider the simple example of the resemblances that ordinal numbers have to cardinal numbers. 
In most of the languages 
this similarity is present in all numerals but the first. 
English presents different patterns for its first three ordinals and cardinals.
%In English the first three numerals present different
%patterns when comparing their ordinal and cardinal counterparts. 
See the examples bellow where we have underlined the
numerals that don't present resemblances between their ordinal and cardinal expressions:
\begin{description}
\item[English:] \underline{one (first)}, \underline{two (second)}, \underline{three (third)}, four (fourth), five (fifth), six (sixth), seven (seventh), eight (eighth), nine (ninth)
\item[German:] \underline{eins (erste)}, zwei (zweite), drei (dritte), vier (vierte), fünf (fünfte), sechs (sechste), sieben (siebte), acht (achte), neun (neunte)
\item[French:] \underline{un (premier)}, deux (deuxième), trois (troisième), quatre (quatrième), cinq (cinquième), six (sixième), sept (septième), huit (huitième), neuf (neuvième)
\item[Hebrew:] \underline{ehad (rishon)}, \underline{shnayim (sheni)}, shlosha (shlishi), \underline{arba'a (revi'i)}, hamisha (hamishi), shisha (shishi), shiv'a (shvi'i), shmona (shmini), tish'a (tshi'i)
\item[Estonian:] \underline{üks (esimene)}, \underline{kaks (teine)}, kolm (kolmas), neli (neljas), viis (viies), kuus (kuues), seitse (seitsmes), kaheksa (kaheksas), üheksa (üheksas)
\end{description}
 There seems to be no explanation for this observation, there is no reason why some languages would behave differently from
 others when regarding numerals names. It also doesn't seem plausible that some cultures have different perception
 with respect to ordering things that could create such discrepancy. It might than be that languages are the way they
 are just by chance, but it is highly compelling to try to find reason for some common behavior.




%%%%%%%%

We structure our thought using our language, but there are certain situations where words lack and still
exist thought, although it might not be represented as a beautiful chain of the language symbols.
\citep{pinker2003} discusses several examples of thoughts that do not appear to be represented in the mind in
anything like verbal language. 


% When we think, we structure our thought using our language, but on the course we may find ourselves looking for a way to express a certain meaning, and it might take awhile until we finally find how to express it in our own language. It suggests we may think without language, but to which degree would it be possible? It is hard to answer whether there is thought without language, but it might be that always when there is thought rises language, even if it has a very simple and rudimentary structure. Whatever sort of language it is, when not exposed to the outer world, might not be understood by the others. That would wide the meaning of language to a mean of representing thoughts, but here our main concern is communication, how meaning may be transmitted from one point to another.

An American author, political activist, and lecturer, named Helen Adams Keller was born in 1880. 
As she was 19 months old, she contracted an illness which might have been scarlet fever or meningitis. 
It left her deaf and blind. Despite all difficulties imposed by her impairments she was taught and today 
she is widely know as the first deaf-blind person to earn a Bachelor of Arts degree. 
``The story of how Keller's teacher, Anne Sullivan, broke through the isolation imposed by a near 
complete lack of language, allowing the girl to blossom as she learned to communicate, has become known worldwide 
through the dramatic depictions of the play and film \textit{The Miracle Worker}''. 
At age 22, Keller published her autobiography, \textit{The Story of My Life} (1903), from which the excerpt bellow are taken from:

\begin{quote}
Have you ever been at sea in a dense fog, when it seemed as if a tangible white darkness shut you in, 
and the great ship, tense and anxious, groped her way toward the shore with plummet and sounding-line, 
and you waited with beating heart for something to happen? I was like that ship before my education began, 
only I was without compass or sounding-line, and had no way of knowing how near the harbour was. 
``Light! give me light!'' was the wordless cry of my soul, and the light of love shone on me in that very hour.

(...)

We walked down the path to the well-house, attracted by the fragrance of the honeysuckle with which it was covered. 
Some one was drawing water and my teacher placed my hand under the spout. As the cool stream gushed over one hand 
she spelled into the other the word water, first slowly, then rapidly. I stood still, my whole attention fixed upon 
the motions of her fingers. Suddenly I felt a misty consciousness as of something forgotten -- a thrill of returning 
thought; and somehow the mystery of language was revealed to me. I knew then that ``w-a-t-e-r'' meant the wonderful 
cool something that was flowing over my hand. That living word awakened my soul, gave it light, hope, joy, set it free! 
There were barriers still, it is true, but barriers that could in time be swept away.
I left the well-house eager to learn. Everything had a name, and each name gave birth to a new thought. 
As we returned to the house every object which I touched seemed to quiver with life. 
That was because I saw everything with the strange, new sight that had come to me.
\end{quote}

``Language is our legacy. It is the main evolutionary contribution of humans, and perhaps the most 
interesting trait that has emerged in the past 500 million years'' \citep{nowak2002}. Understanding the origins 
of our syntactic communication process, how language works and evolves is important to understand
our own legacy. Currently there are many efforts from multidisciplinary fields of study, inquiring
and seeking to understand and better describe language. The study of formal language theory, learning
theory, human psychology and physiology, observations and empirical validations, statistics and
mathematical modeling are some of the aspects taken under consideration when studying 
language as a biological phenomena of communication.
