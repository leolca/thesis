\chapter{Introduction}
Language is a biological, psychological and social process. 
The study of language as a communication process involves insight on these subjects 
and a scientific analysis of data produced as a mean of information transfer. 
Performing a statistical analysis of language means 
to acknowledge its unpredictable nature, the uncertainty intrinsic to it is the 
way in which is it possible to carrying information.
%without which there would be no communication at all. 
Although it has a random nature, 
it holds an order, coordination and structuration that 
imposes an amount of redundancy to the transmitted message.
%it is not like a single roll of a dice, so it 
It is important to characterize the process and understand what variables 
are into play in the communication process.
%influence the choices we make to achieve communication. 
Language is not a process controlled by a single agent, rather it is 
driven by interactions of multiple agents, it is wholly decentralized or distributed 
over all the components of the system.
%We know that 

The patterns on language are important on its usage, learning and efficiency.
A well known example is the word frequency effect on lexical access \citep{whaley1978,grainger90,saly1989}.
Low frequency words require greater effort than high frequency words on recognition task,
leading to a poorer performance on speed and accuracy for the low frequency ones.
Words might be ranked in order of their frequencies of occurrence.
%, and it is known that 
%this frequency can powerfully influence the accuracy with which we hear, read, memorize, 
%associate or use them to build our own speech. 
The lengths of words are not a mere hazard 
but a rational deliberation aiming a thrifty and efficient use of resources. 
The way a language sound system is organized seeks a maximal dissimilarity between stimuli. 
%We ought to believe that there is something else guiding our choices to build up speech utterances, 
%other than the random choices that guide a bunch of monkeys pushing keys on a typewriter. 
The formation of a language is guided by choices, organizing and
structuring this random process. Languages are complex systems which emergence is
a event of central importance in human evolution. Several remarkable features suggest the presence of
a fundamental principle of organization that seems to be common among all languages.
``If a statistical test cannot distinguish rational from random behavior, clearly it cannot 
be used to prove that the behavior is rational. But, conversely, neither can it be used to 
prove that the behavior is random. The argument marches neither forward nor backward''\citep{miller1965}.

The idea proposed by \cite{zipf1949} is that language works seeking the principle of least effort. 
This theory is also applied in different fields such as evolutionary biology. It postulates that
it is a natural behavior to seek a path of low cost.
Although this principle seems reasonable to be applied to all languages and their evolution,
it creates an opposition with other features needed so that a communication processes may be held,
such as variability and distinguishability. This trade-off might be seen as a result of the conflicting
application of the principle of least effort to the speaker and the hearer. Speakers want to 
minimize articulatory effort, use sequences of phones that are easy to pronounce and encourage
brevity and phonological reductions. On the other hand, hearers want to minimize the decoding effort
of uttered speech, it is necessary then to enhance explicitness, clarity and distinguishability.
The hear want to avoid ambiguity in the lexical level, minimizing the effort to understand a sentence.
The speaker will tend to choose the most frequent words, which tend to be the most ambiguous ones
\citep{gernsbacher1994, kohler1986}.
Zipf pointed that this lexical trade-off could explain the pattern of word frequencies,
although no rigorous proof of its validity was ever given \citep{zipf1949}.
 


``Underlying the endless and fascinating idiosyncrasies of the world's languages there are 
uniformities of universal scope. Amid infinite diversity, all languages are, as it were, 
cut from the same pattern. Some interlinguistic similarities and identities have been formalized, 
others not, but working linguistis are in many cases aware of them in some sense and use them as 
guides in their analyses of new languages'' \citep{greenberg1966}.
All languages exhibit two distinguished traits: syntax and symbolic reference
\citep{chomsky1968b, deacon1997}.
As they are always used as a communication system, that uses the same physical medium to 
transport information, uses the same biological apparatus to encode and decode the transmitted
messages and is a mean of social interaction, being shared among a community, they should also 
share other characteristics regarding their constituents parts, structure and usage.


%The process of communication is better when it is possible to transmit information in a fast and robust way. 
%Although this principle seems reasonable to be applied to the whole communication process, 
%we find some individuals, most of them politicians, that insist on going against the flow, 
%usually with a reason. Disregarding those outliers, a better choice to achieve short construction of 
%utterances is to use short building blocks to express the most recurrent words. 

The linguistic analysis of a language is the observation of certain recurring patterns,
their transformation over time and interactions. 
Patterns that occur systematically across natural languages are called linguistic universals. 
An important goal of linguistics is to explain the reason why these patterns emerge so often,
what is also a concern of cognitive studies.
Some approaches might be used to carry out a systematic research and analyse the role of 
these regularities on languages. We are here concerned with a statistical analysis based 
on real world data, through the usage of linguistics corpus, and with computer simulations
of models mimicking language interactions. 
%The only reasonable way to achieve this is through the analysis of real world data, called linguistic corpus. 
%We have to assume that, if there are linguistic rules, they must explain at least a large amount of our data.

It is still unclear what is the nature of language constituent elements, how they are used, organized and
how they change. The phoneme, taken as a mental representation, the basic element of spoken language, 
has been questioned over its status on the study of language \citep{port2007,port2005,port2006}.
%One intriguing question not yet answered is: what are the building blocks of language and how do they organize themselves to build communication? Although there are reasons to believe phonemes are questionable units of language, as pointed out by \cite{port2007,port2005,port2006}, it is still the most accepted and will be adopted here. 
\cite{port2007} argues that ``words are not stored in memory in a way that resembles the abstract, 
phonological code used by alphabetical orthographies or by linguistic analysis''. 
According to him, the memory language works as an exemplar memory, where the information stored is an amalgam 
of auditory codes which include nonlinguistic information. The acceptance and usage of the phonetic model is 
a reflex of our literacy education \citep{port2007,coleman2002}. The assumption of a segmental description of 
speech is important because it guarantees a discrete description at the lower level, what implies discreteness 
at all other levels. All formal linguistic is based on one \textit{a priori} alphabet of discrete tokens.

Some results, pointed by \cite{port2007}, are against this segmental view of language. 
The familiarity with one speaker's voice, improve the speech recognition at approximately 6\%, 
and this improvement is increased slightly as the variability of the others speakers increase. 
\cite{port2007} also argues that richness in dialect variation and language change might not be explained 
when language information is not stored in a detailed form. Another argument is the well-known frequency effect. 
When listening to words in noise, the most frequent words in the language can be more accurately recognized 
then less frequent words. It is also known that ``the frequency of words and phrases can have a major influence 
on speech production in most languages. Typically frequent words suffer greater lenition, that is, 
reduction in articulatory and auditory distinctness, than infrequent words''\citep{port2007}. 

The idea of discrete entities being born from the continuous is an interesting one,
for correspondence can be drawn with the information carried by a continuous energy process.
A discrete system is assumed to involve higher levels of organization.
It was always obvious that spoken language has a continuous substratum, but
it was the major objective of linguistic structuralism to describe phonology
as a system built on discrete entities and logical rules, what would imposes
a discrete structure on a phonetic continuum \citep{chomsky1957}. 
\cite{mandelbrot1954} argues that, at phonological level of language, discreteness is a necessary feature
and there is a necessary relationship between continuous and discrete in linguistic change.
\cite{wilden2001} points that ``digitalization is always necessary when certain boundaries are to be crossed,
boundaries between systems of different \textit{types} or of different \textit{states},
although how these types or boundaries might be operationally defined is unclear''. 

%Although there are some observed data that might not be explained by the usage of a phonetic alphabet as our 
%speech building blocks, as pointed out above (more information may be seen in \citep{port2007}), 
%we will use such a description and keep in mind its drawbacks, that might also leads our analysis to 
%some unexplained points. 


The analysis here proposed consists of using large text databases 
as our language corpora. We intend to analyze the data in different levels
and for this purpose we use pronouncing dictionaries (to perform analysis on the phonemic level),
and syllabification dictionary transcription (to analyze syllables).
We are then able to procure statistical information on the usage of phones,
diphone, triphones, syllables and words in a language.
%and pronouncing 
%dictionaries to procure statistical information on the usage of phones in a language.
Although written and spoken languages present some marked differences, we assume 
it is still reasonable
to estimate the phonological patterns in a spoken language through  %are very similar 
the patterns observed in its written counterpart, when a transcription into the phonemic-level is used.
Spoken language tend to be more ragged, repetitions are more often and the vocabulary
is smaller. Even though these differences exists
we assume the patterns in a spoken language might be, at least coarsely, estimated using written texts.
The focus of our analysis will be on the English language, for that reason we used English text databases
and an English Pronouncing Dictionary. Each text word was transcribed into phonemic 
writing using the Carnegie Mellon University Pronouncing Dictionary. 
Although the words figure in the text within a context, the iterations between the last sounds of a previous word 
and the initial sounds of the following word were neglected.
The analysis of sound structures was restricted to words structures and the text database was used only 
to acquire a statistical estimation on the frequency in which English phones occur.  
The same analysis here proposed may be applied to other languages, 
given a text database and a pronouncing dictionary on this language.
%all it takes is to collect texts, build a large database and use a pronunciation dictionary 
%to transcribe words into phones in that language. 
If languages are organized through a similar approach, it might be possible to find recurrent patterns 
as we analyze different languages. 

%The same analysis described in this text will also be carried out for Portuguese. In order to do so, a large text database has been collect from the newspaper \textit{O Globo} which has an open on-line archive. There are 4 years of published newspaper available, counting approximately 700,000 articles. The process of transcribing text into phones might be carried out through 2 different approaches: (1) using the ASPA database and the text to phone conversion software \citep{SilvaAlmeida2005}; (2) using the rule-based grapheme-phone converter from \cite{SilvaDenilson} which presents an average accuracy of 98\%. With these data at hand, it is possible to derive the same analysis for the Brazilian Portuguese language.

%When no pronouncing dictionary is available, it is still possible to use a text to phone conversion software
%which uses a grapheme-phone conversion rules. But it is necessary to attest the accuracy of the 
%conversions generated by this mean. Some grapheme to phone conversion (G2P) exists for other languages, for example:
%a German G2P is proposed by \cite{Bisani03}; another for Thai is proposed by \cite{Charoenpornsawat};
%for Brazilian Portuguese \citep{SilvaAlmeida2005} and \citep{siravenha2008}.
%The usage of these G2Ps with a text database makes it possible to statistically analyze other languages  
%and to look for regular patterns across different language organization systems.

Although ``languages are simultaneously products of history and entities existing at particular times''\citep{good2008},
both diachronic and synchronic aspects are important to determine what languages are, we focus here on the
synchronic aspects. %and using a timestamped database, we might also achieve diachronic aspects of the language. 
%Future statistical analysis of languages, as the one proposed here, may be used to build a motion
%picture of language change, 
The diachronic approach is also important to investigate since it might clarify 
how languages change and even determine how usage is responsible for these changes. 
Language is a social construct and so it is driven by human society.

The statistical analysis of languages organization may be essential to determine what sound patterns tend to be 
ubiquitous, what are the linguistic universal.
%`natural'. 
As pointed by \cite{mielke2005}, ``phonetically coherent classes like \textipa{/m n N/} and \textipa{/u o O/} 
seem to recur in different languages, while more arbitrary groupings like \textipa{/m n tS/} and 
\textipa{/I P k\super w/} are less common''. What might be explained, according to \cite{mielke2005} 
by two different claims: (1) an innatist, that argues that common classes (sounds patterns) may be described 
by a conjunction of distinctive features; (2) an emergentist, in which common classes are result of a 
common fate. It is also important to understand how these classes are built and used within a language. 
This sort of analysis might be useful to answer questions like that.

One important aspect to understand language organization and evolution is to understand the 
interrelations among the different phones in a language.
The contrasts we make between sounds is an important aspect to define phonetic similarity what 
``is a prerequisite for using it to account for phonological observations''\citep{mielke2005}. 
It is still not clear what might be the grounds to establish such a metric, but it ought to be investigated 
through a statistical analysis of how speech systems are organized and used. 
\cite{mielke2005} pointed that ``what is needed is a similarity metric based on objective measurements of sounds. 
In order to develop such a metric, it is necessary to choose sounds to measure and ways to measure them''.

In the present work we shall analyze some statistical features of the English language, the occurrence 
of patterns and the information content transmitted in a message. Some recurring behaviour are referred to as 
\emph{laws} and described mathematically in the context of Quantitative Linguistics.
Those mathematical descriptions, along with statistics and information theory, 
are used to model language characteristics and inquire on its structures, usage and evolution.



