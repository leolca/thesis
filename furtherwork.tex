\chapter{Further Work}

A study on language phonology, where the focus is on the statistics of language, was presented here. 
%The main idea we intend to develop is that l
Language, as a social tool, is created and driven by society and its changes. 
The way languages are structured regulate their usages and their usage shapes languages structures. 
Languages are then self-organizing feedback systems.
In order to understand how this complex system words, it is also necessary to investigate quantitatively 
the patterns observed in language usage.

Some analysis developed here focuses only on English due to the wide spread usage in phonology study, 
what makes it easy to find tools and references. Here we used a text database to build a list of words 
ranked according to their frequencies of occurrence. This list was later transcribed through 
a pronouncing dictionary, which gives us a ranked list of spoken words represented by their phones. 
It is assumed that the frequency of occurrence of phone, diphone, etc. in a language is similar to 
the frequencies found by means of this ranked list. The occurrence and co-occurrence of phones, 
the occurrence of phones in a word, the phonemic length of words, etc. were analyzed. 
Other questions might still be answered using this approach, and other analysis might also be proposed.

In order to acquire a better insight in this subject, it is necessary a better statistical insight 
on the way languages are structured and used. 
The observation of language changes and interactions are also important in this process to determine 
how languages work. Another approach to a better understanding is through computer simulations. 
As an example, we may cite the experiments developed by \cite{steels97} ``in which robotic agents and 
software agents are set up to originate language and meaning. The experiments test the hypothesis that 
mechanisms for generating complexity commonly found in bio-systems, in particular self-organization, 
co-evolution, and level formation, also may explain the spontaneous formation, adaptation, and growth 
in complexity of language''. \cite{deboer01} also develop similar research, where it is run a computer 
simulation of the emergence of vowel systems in a population of agents. ``The agents (small computer 
programs that operate autonomously) are equipped with a realistic articulatory synthesizer, a model of 
human perception and the ability to imitate and learn sounds they hear. It is shown that due to the 
interactions between the agents and due to self-organization, realistic vowel repertoires emerge. 
This happens under a large number of different parameter settings and therefore seems to be a very robust 
phenomenon. The emerged vowel systems show remarkable similarities with the vowel systems found in human 
languages. It is argued that self-organization probably plays an important role in determining the vowel 
inventories of human languages and that innate predispositions are probably not necessary to explain the 
universal tendencies of human vowel systems''. It seems a promising idea to use computer simulations of 
intelligent agents to show the role self-organization plays on the structuring effect observed on languages. 
Associated with statistical measures of the languages of the world, it is possible to compare the evidences 
and use the statistical information to build better simulations.

The ideas presented up to this point might be used in future research. There are a multitude of possible 
approaches and perspectives still to be analyzed, so it is not the intention to deplete all of them, but 
rather to have a wide view of the possibilities and interacting variables in order to create a better description 
and avoid possible contradictions with aspects not deeply studied.

