\chapter{Language Structure}
``The poet, the philologist, the philosopher, and their traditional associates have ceased
to be the only ones concerned with the structure of human discourse. Increasing numbers
of mathematicians and of engineers have joined them, attracted by technological problems,
formerly ignored or considered trivial, suddenly become so important that they elide traditional
barriers between fields, even those which merely contained intellectual curiosity'' \cite{mandelbrot1965}.
The study of language has become an interdisciplinary field
where contributions from different points of views might be important to draw deeper insights
on the understanding of this communication phenomena. It is a difficult task
to formalize a theory to explain how order is able to emerge from the unpredictability characteristic 
of the human discourse.

To understand the communication process through language it is necessary to build a model to explain how languages work. 
This model should be able to explain how we use language to code information into utterances, how we use our language knowledge 
to interpret an incoming spoken message and retrieve information. Those aspects are under concern 
of phonology. A simple explanation of what regards phonology is to say that it stands for the study of sound structure in language. 
The object of study of phonology is ``an abstract cognitive system dealing with rules in a mental grammar: principles of subconscious 
\textit{thought} as they relate to language sound. (...) the \textit{sounds} which phonology is concerned with are symbolic sounds -- 
they are cognitive abstractions, which represent but are not the same as physical sounds''\citep{odden2005}. 
In the study of phonology we are concerned about what are the abstract representations of language and how are the rules in 
the coding process, from an abstract entity into a physical realization manifested as acoustic waveform, which carries 
intrinsic information of the speech utterance, i.e. information of the message, code and medium.

Not every type of sound may be produced by our phonatory apparatus. Even posed that restriction, the repertoire of speech sounds is enormous (919 different speech sounds were found in UCLA Phonological Segment Inventory Database, UPSID). The sounds of languages are also not equal across the multitude of languages in the World. In German the word for `beautiful' is `schön' (pronounced as \textipa{[S\o:n]}). The vowel \textipa{[\o]} does not exist in English, Portuguese and other languages, but it does in French (the word `jeûne', pronounced as \textipa{[Z\o n]}, meaning noun `fast') and Norwegian (the word `\o l' means `beer' and is pronounced as \textipa{[\o l]}), for example. Not only the sounds differ from language to language, but also the way they might be combined. Certain combinations of sounds are allowed and others are impossible in a language. Some combinations are allowed in certain positions, but not in others. The consonantal cluster \textipa{[ts]} is allowed in English and in German, but it does not happen in Portuguese. In English it may not happen word initially; there is no English word beginning with \textipa{[ts]}, but it may be found in word final position, such as `cats' (\textipa{[k\ae ts]}) and `splits' (\textipa{[splits]}). In German the consonantal cluster is allowed in word final position, as the word `Konferenz' (\textipa{[kOnfe'rEnts]}), meaning `conference'; and it is also allowed word initially, the word for `time' is `Zeit' (\textipa{[tsaIt]}).

The knowledge of those rules is somehow internalized in a native speaker abilities so that (s)he may judge weather or not an unknown word fits the language structure, meaning that this unknown word could be a possible or impossible word in that language. It is usual to find people playing of making new words. 
\begin{quote}
I have also invented some new words. `Confuzzled', which is being confused and puzzled at the same time, `snirt', which is a cross between snow and dirt, and `smushables', which are squashed groceries you find at the bottom of the bag. I have sent a letter to the Oxford Dictionary people asking them to include my words but I have not heard back. -- from the movie `Mary and Max' (2009) 
\end{quote}
It would not be a surprise to find new words self-invented, like the ones in the example above, in a dictionary someday, just as some self-invented words were already incorporated to the dictionaries. Some examples that have been incorporated to English are `robotics' (meaning the technology of design, construction, and operation of robots) invented by Isaac Asimov\footnote{Isaak Yudovich Ozimov was born in Petrovichi, Byelorussian SSR, between October 4, 1919 and January 2, 1920. His family immigrated to the United States when he was three years old. He lived in the Brooklyn, New York. He studied in the Seth Low Junior College for two years and then in the Columbia University, where he graduated in 1939. After he made a Ph.D. in biochemistry in the same institution. In his life, Asimov wrote and edited more than 500 books and is widely known by his science-fiction writings.} in his 1941 science fiction short-story `Liar!'; and `warp speed' (meaning the highest possible speed), term first used in the 1960s by the series \textit{Star Trek} and incorporated in the dictionaries in 1979. The term `snirt', although not yet incorporated to many dictionaries, is already reported by more dynamic dictionaries like the \textit{Wikitionary} and is commonly used by people. In German, that is an analytical language, this kind of dirty snow is named `Schneematsch' (comes from `Schnee' (snow) + `Matsch' (mud)).
 
``In particular, the frequency with which individual words or sequences of words are used and the frequency with which certain patterns recur in a language affect the nature of mental representation and in some cases the actual phonetic shape of words''\citep{bybee2003}. The structuralism provided a way to analyze the speech continuum into a sequence of units, and these units into features; establishing hierarchical relations between them and organizing the speech knowledge into different levels of a grammar built of phonology, morphology, syntax, and semantics. The way language is used as a social-interactional tool and the frequency of occurrence of certain patterns are determinant factors to explain the language phenomenon. ``It is certainly possible that the way language is used affects the way it is represented cognitively, and thus the way it is structured''\citep{bybee2003}.

``The proposal that frequency of use affects representation suggests a very different view of lexical storage 
and its interaction with other aspects of the grammar or phonology than that assumed in most current theories. 
Structuralist and generative theories assume that the lexicon is a static list, and that neither the rules nor 
the lexical forms of a language are changed at all by instances of use''\citep{bybee2003}. 
It is important to have a language model capable of explaining some language usage facts as, for example, 
the fact that the rate and extent of a phonological change is directly affected by the frequency of the 
involved items in the lexicon. The way phonological rules and phonological representations are stated should 
consider those aspects of languages. A good conceptualization of phonology shall not forget that, 
as part of the procedure for producing and understanding language, the phonological properties of a language 
must be highly associated with the vocal tract and its usage. \cite{bybee2003} proposes that language is governed 
by cognitive and psychological processes and principles which are not language specific, but the same that govern 
other aspects of human cognitive and social behavior.

We believe language might be modelled as a self-organized complex system which is made up of many interacting parts.
Each individual part, called `agent', interact in a simple way with its neighbours, what might lead to large-scale
behaviours that might not be predicted from the knowledge only of the behaviour of the individual parts.  
What we observe as phonological rules and patterns of language might be understood as these collective behaviours 
that emerge from this simple process of interacting agents in a complex system. The interactions between agents might
be taken in different forms. The famous predator-prey approach is used by \cite{wang2004} to model lexical diffusion.
Such computational studies of language emergence provide a valuable way to study language evolution. This field of
study started with \cite{hurford89} simulation model on lexical emergence and acquisition. Hurford considered 
``three conceivable strategies for acquiring the basis of communicative behaviour, here labelled the Imitator, 
Calculator, and Saussurean strategies'' \citep{hurford89}. The first strategy consists of imitating others agents
when they refer to certain objects. The second approach consist of reinforcing the usage of a certain utterance 
when the neighbour agents respond positively. According to Hurford, the better approach is the last one, the
Saussurean strategy, in which the agents copy the patterns produced by nearby individuals, but make their perception
consistent with their own production. ``Many models of language evolution have adopted the agent-base simulation
paradigm'' \citep{wang2004}. All of them need real world quantitative data to use on the models and to validate
the outcomes. On the following sections we intend to quantitatively analyze some natural language data.


